% vim: autowrite
\documentclass[11pt, a4paper, landscape]{scrartcl}

\usepackage{cheatsheet} % See cheatsheet.sty
\usepackage{nimbusmononarrow} % Required for tt font type
\usepackage[utf8]{inputenc} % Required for inputting international characters
\usepackage[T1]{fontenc} % Output font encoding for international characters
\usepackage{graphicx} % Required for including images
\usepackage{fdsymbol} % Required for some math symbols, like \lAngle
\usepackage[dvipsnames]{xcolor} % Required for source-highlight coloured output
\usepackage{textcomp} % Required for \textquotesingle

\usepackage{listings} % Required for inserting code snippets

\definecolor{DarkGreen}{rgb}{0.0,0.4,0.0} % Comment color
\definecolor{highlight}{RGB}{255,251,204} % Code highlight color

\lstloadlanguages{Perl,Java,[Sharp]C,Python,bash}
\lstdefinestyle{SourceCode}{
backgroundcolor=\color{highlight},
basicstyle=\footnotesize\ttfamily,
breakatwhitespace=false,
breaklines=true, % Automatic line breaking (prevents code from protruding outside the box)
captionpos=b,
commentstyle=\color{DarkGreen}, % Style of comments within the code
deletekeywords={}, % If you want to delete any keywords from the current language separate them by commas
%escapeinside={\%}, % This allows you to escape to LaTeX using the character in the bracket
firstnumber=1, % Line numbers begin at line 1
frame=single, % Frame around the code box, value can be: none, leftline, topline, bottomline, lines, single, shadowbox
frameround=tttt, % Rounds the corners of the frame for the top left, top right, bottom left and bottom right positions
keywordstyle=\color{Blue}\bf, % Functions are bold and blue
morekeywords={}, % Add any functions no included by default here separated by commas
numbers=none, % Location of line numbers, can take the values of: none, left, right
numbersep=10pt, % Distance of line numbers from the code box
numberstyle=\tiny\color{Gray}, % Style used for line numbers
rulecolor=\color{black}, % Frame border color
showstringspaces=false, % Don't put marks in string spaces
showtabs=false, % Display tabs in the code as lines
stepnumber=5, % The step distance between line numbers, i.e. how often will lines be numbered
stringstyle=\color{Purple}, % Strings are purple
tabsize=2, % Number of spaces per tab in the code
}
\lstdefinestyle{Perl}{language=Perl,style=SourceCode}
\lstdefinestyle{Java}{language=Perl,style=SourceCode,morecomment=[l]{//}}
\lstdefinestyle{CSharp}{language=[Sharp]C,style=SourceCode}
\lstdefinestyle{Python}{language=Python,style=SourceCode}
\lstdefinestyle{Bash}{language=bash,style=SourceCode,morekeywords={perl}}

\usepackage[% This block contains information used to annotate the PDF
	colorlinks=true,
	urlcolor=blue,
	pdftitle={Regular Expressions},
	pdfauthor={Gastón Simone},
	pdfsubject={Regular Expressions cheatsheet},
	pdfkeywords={Regular Expressions, Cheatsheet}
]{hyperref}

\newcommand{\regex}[1]{\texttt{#1}}
\newcommand{\cregex}[1]{\colorbox{gray!30}{\regex{#1}}}
\renewcommand{\cmdleft}[1]{\cregex{#1}}
\newcommand{\code}[1]{\texttt{#1}}
\newcommand{\bs}{\textbackslash}
\newcommand{\re}{\textit{re}}
\newcommand{\dnet}{{.}Net}
\newcommand{\insertcode}[2]{\lstinputlisting[label=#2,style=#1]{#2}}

\newcommand{\reshortexample}[1]{\cregex{\input{#1}}}
\newcommand{\relongexample}[1]{\regex{\input{#1}}}

\begin{document}

\begin{cheatsheet}{Regular Expressions Basics}

\begin{col1}

\re\ = \emph{regular expression}.\\
\emph{greedy} = always favour matching\\
\emph{lazy} = always favour not matching

\sectiontitle{Any character}

\command{.}{(dot) Any character, except new line}
\command{\bs.}{Literal dot}

\sectiontitle{Character classes (custom)}

\command{[\dots]}{Any character listed}
\command{[\^{}\dots]}{Any character \emph{not} listed}
\command{[a-z]}{Range, any character from a to z}
\command{[-\dots]}{To include dash (-), add it first}

\sectiontitle{Boundaries}

\command{\^{}, \$}{[Start, End] of line}
\command{\bs{}A, \bs{}Z, \bs{}z}{[Start, End, End] of string}
The difference between these occurs in multi-line mode.\\

\command{\bs{}b}{Word boundary (Perl, Java, \dnet)}
\command{\bs{}B}{Not word boundary}
\command{\bs{}<,\bs{}>}{[Start, End] of word\textsuperscript{$\dagger$} (Vim)}
\command{\bs{}G}{Match starts where the previous match ended}
\command{\bs{}Q\emph{text}\bs{}E}{Literal \emph{text} inside regex}

$\dagger$ Perl/Java/\dnet:\\
\mbox{}\ \ \ \ \regex{\bs{}<} $\to$ \regex{(?<!\bs{}w)(?=\bs{}w)}\\
\mbox{}\ \ \ \ \regex{\bs{}>} $\to$ \regex{(?<=\bs{}w)(?!\bs{}w)}\\
(see \emph{Lookaround} in next page).

\end{col1}

\begin{col2}

\sectiontitle{Grouping and Backreferences}

\command{(\re)}{Group \re\ as a unit (Perl, Java, \dnet)}
\command{\bs(\re\bs)}{Vim without \emph{very magic}}
\command{\bs1,\dots,\bs9}{$[1..9]$-th matched \regex{()}}

\sectiontitle{Substitutions}

\command{\regex{\$1},\dots,\regex{\$9}}{$[1..9]$-th matched \regex{()}}
\command{\regex{\$+}}{Last matched \regex{()}}
\command{\$\&, \&}{Entire matched string (Perl, Vim)}
\command{\$\`{}, \$\textquotesingle{}}{Text [before, after] match}
\command{\$\_}{Entire input text}

\sectiontitle{Quantifiers (greedy)}

(Perl/Java/\dnet, Vim without \emph{very magic}):\\
\command{\re?, \re\bs?}{\re\ optional (0 or 1)}
\command{\re*}{Any quantity of \re\ (0 or more)}
\command{\re+, \re\bs+}{At least one of \re\ (1 or more)}
\command{\re\{n\}, \re\bs\{n\}}{\re\ exactly $n$ times}
\command{\re\{n,\}, \re\bs\{n,\}}{\re\ at least $n$ times}
\command{\re\{n,m\}, \re\bs\{n,m\}}{\re\ at least $n$ and at most $m$ times}

\sectiontitle{Quantifiers (lazy)}

(Perl/Java/\dnet, Vim without \emph{very magic}):\\
\command{\re??, \re\bs\{-,1\}}{\re\ optional (0 or 1)}
\command{\re*?, \re\bs\{-\}}{Any quantity of \re\ (0 or more)}
\command{\re+?, \re\bs\{-1,\}}{At least one of \re\ (1 or more)}
\command{\re\{n\}?, \re\bs\{-n\}}{\re\ exactly $n$ times}
\command{\re\{n,\}?, \re\bs\{-n,\}}{\re\ at least $n$ times}
\command{\re\{n,m\}?, \re\bs\{-n,m\}}{\re\ between $n$ and $m$ times}

\end{col2}

\begin{col3}

\sectiontitle{Alternation (branching)}

\command{\re|\re}{\re\ or \re\ (Perl/Java/\dnet)}
\command{\re\bs|\re}{\re\ or \re\ (Vim without \emph{very magic})}
\emph{Note:} In NFA engines alternatives are tried in order of appearance. So
placing them in probability order prevents backtracking, improving
efficiency.\\

\sectiontitle{Shorthands}

\command{\bs{}s}{White space (space, tab, etc.)\textsuperscript{$\dagger$}}
\command{\bs{}S}{Not \regex{\bs{}s}}
\command{\bs{}d}{Digit (\regex{[0-9]})\textsuperscript{$\dagger$}}
\command{\bs{}D}{Not \regex{\bs{}d}}
\command{\bs{}w}{Word char (\regex{[a-zA-Z0-9\_]})\textsuperscript{$\dagger$}}
\command{\bs{}W}{Not \regex{\bs{}w}}
\command{\bs{}t, \bs{}n}{Tab, New line}
\begin{cheatnote}
$\dagger$ Some flavours also recognise special Unicode characters in the same character group.\\
In Vim \regex{\bs\_s} also includes new line.\\
\end{cheatnote}

\sectiontitle{Modes and Flags}

\command{i}{Ignore case\textsuperscript{$\dagger$} (Perl/Vim)}
\command{m}{Multi-line mode (Perl)}
\command{s}{Dot matches new line\textsuperscript{$\ddagger$} (Perl)}
\command{x}{Ignore white space\textsuperscript{\textsection} (Perl)}
\command{g}{Apply substitution in all occurrences (Perl/Vim)}

\begin{cheatnote}
$\dagger$ Java: \code{Pattern.CASE\_INSENSITIVE},\\
\dnet: \code{RegexOptions.IgnoreCase}.
\end{cheatnote}
$\ddagger$ In Vim, use \regex{\bs\_.} to make dot match new lines.\\
\begin{cheatnote}
\textsection\ Java: \code{Pattern.COMMENTS},\\
\dnet: \code{RegexOptions.IgnorePatternWhiteSpace}.
\end{cheatnote}

\end{col3}
\end{cheatsheet}

\newpage

\begin{cheatsheet}{Regular Expressions Advanced}

\begin{col1}

\sectiontitle{Lookaround}

\emph{Lookahead} = Check if \re\ matches at current position, but do not consume.\\
\emph{Lookbehind} = Check if \re\ matches just before what follows, but do not consume.\\
\emph{Negative} = Check if \re\ does \emph{not} match.\\

(Perl/Java/\dnet, Vim):\\
\command{(?=\re), \re\bs{}@=}{Lookahead}
\command{(?!\re), \re\bs{}@!}{Negative lookahead}
\command{(?<=\re), \re\bs{}@<=}{Lookbehind}
\command{(?<!\re), \re\bs{}@<!}{Negative lookbehind}

\sectiontitle{Non-capturing parentheses and comments}

Group as a unit, but do not capture for backreferencing.\\
\command{(?:\re)}{Perl, Java, \dnet}
\command{\bs\%(\re\bs)}{Vim}

\command{(?\#\emph{free text})}{Comment inside regex}

\sectiontitle{Branch reset}

\command{(?|\re)}{Restart group numbering for each branch in \re}

Not available in Vim.\\

\end{col1}

\begin{col2}

\sectiontitle{Named captures}

\command{(?<\emph{name}>\re)}{Capture \re\ under \emph{name}}
\command{\bs{}k<\emph{name}>}{Named backreference}
\command{\$+\{\emph{name}\}}{Named substitution (Perl)}
\command{\$\{\emph{name}\}}{Named substitution (\dnet)}

Not available in Vim.\\

\sectiontitle{Atomic grouping}

Match \re\ without retry. In a NFA engine, discard all the possible backtracking
states for the enclosed \re.\\

\command{(?>\re)}{Perl/Java/\dnet}
\command{\re\bs{}@>}{Vim}

\sectiontitle{Possessive quantifiers}

\command{\re?+}{\regex{(?>\re?)}}
\command{\re*+}{\regex{(?>\re*)}}
\command{\re++}{\regex{(?>\re+)}}
\command{\re\{n,m\}+}{\regex{(?>\re\{n,m\})}}

Not available in Vim (but can be done with atomic grouping).\\

\sectiontitle{End of previous match}

\command{\bs{}G}{Matches where the previous match ended}
Useful when using the \regex{g} mode.\\
In Perl, use \code{pos(\$str)} to know the current location where
\regex{\bs{}G} would match in \code{\$str}.\\
Not available in Vim.

\end{col2}

\begin{col3}

\sectiontitle{Conditional expressions}

If \emph{cond} then \re1, else \re2. \re2 is optional.\\

Perl/Java/\dnet:\\
\command{(?(\emph{cond})\re)}{Without \emph{else} \re}
\command{(?(\emph{cond})\re1|\re2)}{With \emph{else} \re}
\command{(?(\emph{n})$\cdots$)}{\emph{cond} = $n$-th \regex{()} matched}

\emph{cond} can be a \emph{lookaround} expression (see the examples).\\
Not available in Vim.\\

\sectiontitle{Dynamic regex and embedded code}

\command{(??\{\emph{code}\})}{Match \re\ built by \emph{code} here}
\command{(?\{\emph{code}\})}{\emph{code} that does anything}
Useful for debugging regex by \texttt{print}ing.\\

Not available in Vim.\\

\sectiontitle{Recursive expressions}

\command{(?R)}{Repeat entire \re\ here}
\command{(?R\emph{n})}{Repeat \re\ captured under group $n$ here}

Not available in Vim.\\

\sectiontitle{Mode modifiers}

(Perl/Java/\dnet, Vim):\\
\command{(?i), \bs{}c}{Turn case-insensitive on}
\command{(?-i), \bs{}C}{Turn case-insensitive off}
\command{(?i:\re)}{Be case insensitive for \re}

\end{col3}

\end{cheatsheet}

\newpage

\begin{cheatsheet}{Regular Expressions and Characters}

\begin{col1}

\sectiontitle{Unicode properties}

Perl/Java/\dnet:\\
\command{\bs{}p\{L\}}{Letters}
\command{\bs{}p\{M\}}{Accent marks, etc.}
\command{\bs{}p\{Z\}}{Spaces, etc.}
\command{\bs{}p\{S\}}{Dingbats and symbols}
\command{\bs{}p\{N\}}{Numeric characters}
\command{\bs{}p\{P\}}{Punctuation characters}
\command{\bs{}p\{C\}}{Everything else}

\sectiontitle{Unicode sub-properties}

\command{\bs{}p\{Ll\}}{Lower-case letters}
\command{\bs{}p\{Lu\}}{Upper-case letters}
\command{\bs{}p\{Mn\}}{Non-spacing mark (accents, \dots)}
\command{\bs{}p\{Sm\}}{Math symbol ($-$,$+$,$\div$,$\leq$,\dots)}
\command{\bs{}p\{Sc\}}{Currency symbol (\textyen,\textcent,\texteuro,\textdollar,\textsterling,\dots)}
\command{\bs{}p\{Nd\}}{Decimal digit}
\command{\bs{}p\{Nl\}}{Letter number (mostly Roman numerals)}
\command{\bs{}p\{Pd\}}{Dash punctuation}
\command{\bs{}p\{Ps\}}{Open punctuation ([, \{, $\lAngle$, \dots)}
\command{\bs{}p\{Pe\}}{Close punctuation (], \}, $\rAngle$, \dots)}
\command{\bs{}p\{Cc\}}{ASCII and Latin-1 control characters (\texttt{TAB}, \texttt{CR}, \texttt{LF}, \dots)}

\sectiontitle{Unicode negated properties}

\command{\bs{}P\{$\cdots$\}}{Perl/Java/\dnet}
\command{\bs{}p\{\^{}$\cdots$\}}{Perl}

\end{col1}

\begin{col2}

\sectiontitle{Unicode blocks and scripts}

\command{\bs{}p\{InCyrillic\}}{Cyrillic character (Perl, Java)}
\command{\bs{}p\{IsCyrillic\}}{Cyrillic character (\dnet)}
\command{\bs{}p\{Latin\}}{Latin character}
\command{\bs{}p\{Greek\}}{Greek character}
\command{\bs{}p\{Hebrew\}}{Hebrew character}

\sectiontitle{POSIX character classes}

\command{[:alnum:]}{Alphanumeric characters}
\command{[:alpha:]}{Alphabetic characters}
\command{[:blank:]}{Space and tab}
\command{[:cntrl:]}{Control characters}
\command{[:digit:]}{Numeric characters}
\command{[:graph:]}{Non-blank characters}
\command{[:lower:]}{Lower-case alphabetic characters}
\command{[:print:]}{\regex{[:graph:]} and the space character}
\command{[:punct:]}{Punctuation characters}
\command{[:space:]}{All whitespace characters}
\command{[:upper:]}{Upper-case alphabetic characters}
\command{[:xdigit:]}{Hexadecimal digits (\regex{[0-9a-fA-F]})}

\sectiontitle{POSIX collating sequences}

\command{[.span-ll.]}{Match ``ll'' as single character}
\command{[.ch.]}{Match ``ch'' as single character}
\command{[.eszet.]}{Match ``ss'' (``ß'') as single character}

\sectiontitle{POSIX character equivalents}

\command{[[=a=][=n=]]}{Character \emph{equivalents} for \emph{a} and \emph{n} (a, á, à, ä, \dots, n, ñ, \dots)}

\end{col2}

\begin{col3}

\sectiontitle{Special characters and other shorthands}

\command{\bs\textit{char}}{Literal \textit{char}}
\command{\bs{}t}{Tab (\texttt{HT, TAB})}
\command{\bs{}n}{New line (\texttt{LF, NL})}
\command{\bs{}r}{Carriage return (\texttt{CR})}
\command{\bs{}f}{Form feed (\texttt{FF})}
\command{\bs{}a}{Alarm (\texttt{BEL})}
\command{\bs{}e}{Escape (think troff) (\texttt{ESC})}
\command{\$\$}{Literal \$ (\dnet)}
\command{\bs{}l}{Lower-case next char}
\command{\bs{}u}{Upper-case next char}
\command{\bs{}L\emph{text}\bs{}E}{Lower-case \emph{text}}
\command{\bs{}U\emph{text}\bs{}E}{Upper-case \emph{text}}
\command{\bs\emph{num}}{Octal escape}
\command{\bs{}x\emph{num}, \bs{}x\{\emph{num}\}}{Hex escape}
\command{\bs{}u\emph{num}, \bs{}U\emph{num}}{Unicode escape}
\command{\bs{}c\emph{x}, \bs{}c\emph{X}}{\texttt{CTRL-\emph{X}}}
\command{\bs{}N\{\emph{U+263D}\}}{Unicode character}
\command{\bs{}N\{\emph{name}\}}{Character by Unicode \emph{name}}

\sectiontitle{Class set operations}

\command{[[a-z]\&\&[\^{}aeiou]]}{Class 1 except class 2 (Java)}
\command{[[a-z]-[aeiou]]}{Class 1 except class 2 (\dnet)}

\end{col3}

\end{cheatsheet}

\newpage

\begin{cheatsheet}{Regular Expressions Efficiency}

\begin{col1}

\sectiontitle{General guidelines}
\begin{itemize}
	\item Avoid recompiling the \re.
	\item Use non-capturing parentheses (but check they are indeed faster in
		your case!).
	\item Split into multiple \re's (or literal text search) if
		it can make the process faster.
	\item Use leading anchors (\cregex{\^{}}, \cregex{\bs{}A}, \cregex{\$},
		\cregex{\bs{}Z}, etc.) to reduce the number of locations where the
		\re{} is evaluated.
	\item Use atomic grouping and possessive quantifiers to avoid unnecessary
		backtracking (especially important in non-matching cases).
	\item Put the most-likely alternatives first (only affects traditional NFA
		engines).
\end{itemize}

\sectiontitle{Lazy quantifiers may be slower than greedy ones}

Because they must jump between checking what the quantifier controls with
checking what comes after. Example: \cregex{"(.*?)"} to match a double-quoted
string. Compare with the example in the next page.

\end{col1}

\begin{col2}

Notation: $worse\to better$

\sectiontitle{Expose literal text}

\cregex{this|that} $\to$ \cregex{th(?:is|at)}.\\
\cregex{-\{5,7\}} $\to$ \cregex{------\{0,2\}}.\\
The engine can use a fast substring search (like the
\href{https://en.wikipedia.org/wiki/Boyer–Moore_string_search_algorithm}{Boyer-Moore}
or the Hume-Sunday algorithms) to find the literal text (i.e. \cregex{th}) and
then match the rest of the \re.

\sectiontitle{Expose anchors}

\cregex{\^{}this|\^{}that} $\to$ \cregex{\^{}(?:this|that)}.\\
\cregex{abc\$|123\$} $\to$ \cregex{(?:abc|123)\$} (only affects Perl).\\
\re{} only tried at places where the anchor (i.e. \cregex{\^{}}) matches.

\end{col2}

\begin{col3}

\sectiontitle{Loop unrolling}

Useful to optimise pattern with the following structure:
\cregex{(normal|special)*}. This can be changed to:\\
\cregex{opening normal*(?:special normal*)* closing},\\
where \regex{opening}, \regex{normal}, \regex{special} and \regex{closing} are
all regular expressions. It can be used to optimise any number of alternatives,
like in \cregex{(\re{}1|\re{}2|\dots)*}.\\
\textbf{Notes:}
\begin{enumerate}
	\item \regex{normal} should be the most common case
	\item the start of \regex{normal} and \regex{special} must not match at the
		same location.
	\item \regex{special} must not match the \emph{empty string}.
	\item \regex{special} must be atomic.
\end{enumerate}

\end{col3}

\end{cheatsheet}

\newpage

\begin{cheatsheet}{Regular Expressions Examples 1}

\begin{col1}

\sectiontitle{Match quoted string with escaped quotes}

\reshortexample{./quoted_string.tex}\\
\emph{Note:} Possessive quantifiers prevent very long execution times (due to
the nested quantifiers \cregex{+} and \cregex{*}) when there is no matching.\\
Loop-unrolling version:\\
\reshortexample{./quoted_string_lu.tex}\\

\sectiontitle{Conditional expressions}

\begin{itemize}
	\item Backreference as condition: Match a word optionally wrapped in \code{<$\cdots$>}:\\
	\reshortexample{./cond_backref.tex}
	\item Lookaround as condition: Check for number only if prefixed with ``\code{NUM:}'':\\
	\reshortexample{./cond_lookaround.tex}
\end{itemize}

\sectiontitle{Branch reset}

Always capture the number under \cregex{\bs{}1} (or \cregex{\$1}):\\
\reshortexample{./branch_reset.tex}

\sectiontitle{Fix floating-point rounding problems}

Use at most $3$ decimal places. Change numbers like $3.27600000002828$ into
$3.276$; or $4.120000000034$ into $4.12$:\\
\reshortexample{./fixfloat.tex}\\

\end{col1}

\begin{col2}

\sectiontitle{Extract file name from path}

Perl:\\
\relongexample{./extract_filename.tex}\\

\sectiontitle{Adding thousand separators to a number}

Using lookaround:\\
\reshortexample{./thousandsep_1.tex}\\

Without using lookaround (in Perl):\\
\relongexample{./thousandsep_2.tex}\\

\sectiontitle{Match continuation line}

Match a single ``logical'' line split into multiple lines by adding \cregex{\bs}
at the end of each split. Example:\\
\texttt{%
\mbox{}\ \ SRC=a.c b.c c.c \bs\\
\mbox{}\ \ \ \ \ \ d.c e.c}

\reshortexample{./cont_line.tex}\\
Loop-unrolling version:\\
\relongexample{./cont_line_lu.tex}

\end{col2}

\begin{col3}

\sectiontitle{Match a date (month and day)}

\relongexample{./date.tex}\\

\sectiontitle{Match time (am/pm)}

\reshortexample{./time12.tex}\\

\sectiontitle{Match time (24 hours)}

\reshortexample{./time24_1.tex}\\
or\\
\reshortexample{./time24_2.tex}\\

\end{col3}

\end{cheatsheet}

\newpage

\begin{cheatsheet}{Regular Expressions Examples 2}

\begin{col1}

\sectiontitle{Parse a CSV file}

This regex matches each field in a CSV line, supporting fields with or without
surrounding double-quotes. In the former case, a double quote is represented by
a pair of double quotes:\\
\relongexample{./parse_csv.tex}\\
The line with the ``slow'' comment can be made faster using loop unrolling,
with \regex{normal} = \cregex{[\^{}"]} and \regex{special} = \cregex{""}:
\reshortexample{./parse_csv_lu.tex}\\

\sectiontitle{Match IP address}

Using Perl regex object for clarity:\\
\relongexample{./ip.tex}\\

\sectiontitle{Match an email address}

\relongexample{./email_addr.tex}

\end{col1}

\begin{col2}

\sectiontitle{Match a URL}

\relongexample{./url.tex}\\

\sectiontitle{Match closing XML tag}

Using lazy quantifier:
\reshortexample{./matchclosetag_lazy.tex}\\
Using greedy quantifier:
\reshortexample{./matchclosetag_greedy.tex}\\
Loop-unrolling with
\regex{normal} = \cregex{[\^{}<]} and
\regex{special} = \cregex{(?!</?B>)<}:\\
\relongexample{./matchclosetag_lu.tex}\\

\end{col2}

\begin{col3}

\sectiontitle{Balanced parentheses (static)}

Match up to $n$ levels of nested parentheses:\\
\relongexample{./balanced_paren_static.tex}\\

\sectiontitle{Balanced parentheses (dynamic)}

Match a function call with balanced parentheses:\\
\relongexample{./balanced_paren_dynamic.tex}\\

\end{col3}

\end{cheatsheet}

\newpage

\begin{cheatsheet}{Regular Expressions Usage in Programming Languages and Tools}

\begin{col1}

\sectiontitle{Perl}

\insertcode{Perl}{./src_examples/finddbl.pl}

\end{col1}

\begin{col2}

\sectiontitle{Java}

\insertcode{Java}{./src_examples/example.java}
\mbox{}% Used to leave some vertical space

\sectiontitle{Python}

\insertcode{Python}{./src_examples/example.py}

\end{col2}

\begin{col3}

\sectiontitle{\dnet\ (C\#)}

\insertcode{CSharp}{./src_examples/example.cs}
\mbox{}% Used to leave some vertical space

\sectiontitle{Automated editing}

\insertcode{Bash}{./src_examples/commands.sh}

\end{col3}

\end{cheatsheet}

\end{document}
